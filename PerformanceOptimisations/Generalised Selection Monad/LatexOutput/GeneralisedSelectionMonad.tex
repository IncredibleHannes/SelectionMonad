% This is samplepaper.tex, a sample chapter demonstrating the
% LLNCS macro package for Springer Computer Science proceedings;
% Version 2.20 of 2017/10/04
%
\documentclass[runningheads]{llncs}

\usepackage{listings}
\usepackage[dvipsnames]{xcolor}
\usepackage{textcomp}
\usepackage{amsmath}
\usepackage{changepage}


\usepackage[english]{babel}

\definecolor{darkgray}{rgb}{.4,.4,.4}

\providecommand{\tightlist}{%
  \setlength{\itemsep}{0pt}\setlength{\parskip}{0pt}}

\lstnewenvironment{code}{\lstset{language=Haskell}}{}
\lstnewenvironment{haskell}{\lstset{language=Haskell}}{}

\newcommand{\blue}[1]{\textcolor{blue}{#1}}
\newcommand{\violet}[1]{\textcolor{violet}{#1}}
\setlength\parindent{0pt}

\newcommand{\equals}[1]{=\quad \{\text{ #1 }\}}

\newenvironment{reasoning}{\begin{adjustwidth}{0.2in}{}}{\end{adjustwidth}}
\newcommand{\ind}[1]{\begin{adjustwidth}{0.15in}{}\begin{math}#1\end{math}\end{adjustwidth}}

\lstdefinelanguage{Haskell}{
  morekeywords={if, then, else, where, forall, deriving,let,in,do},
  emph={[2]Ord, J, K, Decision, GK, Bool, Int, Char, Show,String,ghci}, emphstyle={[2]\color{Mahogany}},
  emph={[3]True,False,Wall,Street}, emphstyle={[3]\color{Brown}},
  ndkeywords={data,type}, 
  sensitive=true,
  comment=[l]{--},
  morecomment=[s]{\{\{--}{--\}\}},
  morecomment=[s]{\{\{}{\}\}},
  string=[b]",
  morestring=[s]{"}{"},
  morestring=[s]{\{\{}{\}\}},
  literate= {(}{{\blue{(}}}1
            {)}{{\blue{)}}}1
            {[}{{\blue{[}}}1
            {]}{{\blue{]}}}1
            {:}{{\violet{:}}}1
            {>}{{\violet{>}}}1
            {|}{{\violet{|}}}1
            {=}{{\violet{=}}}1
            {,}{{\violet{,}}}1
            {-}{{\violet{-}}}1
            {+}{{\violet{+}}}1
            {.}{{\violet{.}}}1
}

\lstset{
    basicstyle=\small\ttfamily\linespread{2},
    breaklines=false,
    columns=space-flexibl,
    commentstyle=\color[rgb]{0.127,0.427,0.514}\ttfamily\itshape,
    escapechar=@,
    extendedchars=true,
    identifierstyle=\color{black},
    inputencoding=latin1,
    xleftmargin=.2in,
    keywordstyle=\color{teal}\bfseries,
    language=Haskell,
    ndkeywordstyle=\color{blue}\bfseries,
    numbers=none,
    prebreak = \raisebox{0ex}[0ex][0ex]{\ensuremath{\hookleftarrow}},
    showstringspaces=false,
    stringstyle=\color[rgb]{0.639,0.082,0.082}\ttfamily,
    upquote=true,
}


\newcommand{\ignore}[1]{}


\begin{document}
\title{Towards a more efficient Selection Monad}

\author{
    Johannes Hartmann\inst{1} \and 
    Tom Schrijvers\inst{2}\and 
    Jeremy Gibbons\inst{1}
}
%
\authorrunning{J. Hartmann, T. Schrijvers, J. Gibbons}

\institute{
    University of Oxford, Department of Computer Science, UK
    \email{firstname.lastname@cs.ox.ac.uk}\and 
    KU Leuven, Department of Computer Science, Belgium,
    \email{tom.schrijvers@kuleuven.be}
}

%
\maketitle              % typeset the header of the contribution
%
\begin{abstract}
This paper explores a novel approach to selection functions through the
introduction of a generalized selection monad. The foundation is laid
with the conventional selection monad \(J\), defined as
\((A \rightarrow R) \rightarrow A\), which employs a pair function to
compute new selection functions. However, inefficiencies in the original
pair function are identified. To address these issues, a specialized
type \(K\) is introduced, and its isomorphism to \(J\) is demonstrated.
The paper further generalizes the \(K\) type to \(GK\), where
performance improvements and enhanced intuitive usability are observed.
The embedding between \(J\) to \(GK\) is established, offering a more
efficient and expressive alternative to the well established \(J\) type
for selection functions. The findings emphasize the advantages of the
generalized selection monad and its applicability in diverse scenarios,
paving the way for further exploration and optimization.

\keywords{Selection monad  \and Functional programming \and Algorithm design \and 
Performance Optimisation \and Monads.}
\end{abstract}
%
%
%
\ignore{

> {-# LANGUAGE ImpredicativeTypes #-}
> {-# LANGUAGE ScopedTypeVariables #-}

> import Prelude hiding ((>>=), return, pure, (<*>), fmap, sequence, pred)
> import Data.Function (on)
> import Data.List
  import GHC.Base (TVar#)
  
}

\section{Introduction to the Selection Monad
J}\label{introduction-to-the-selection-monad-j}

The selection monad, initially introduced by Paulo Oliva and Martin
Escardo \cite{escardo2010selection}, serves as a valuable tool for
modeling selection-based algorithms in functional programming. Widely
explored in the context of sequential games
\cite{escardo2010sequential}, it has been applied to compute solutions
for games with perfect information and has found applications in logic
and proof theory through the Double-Negation Theorem and the Tychonoff
Theorem \cite{escardo2010sequential}. Additionally, it has been
effectively employed in modeling greedy algorithms
\cite{hartmann2021greedyselection}. These diverse applications of the
selection monad heavily rely on its monadic behavior, particularly
emphasizing the use of the \(sequence\) function for monads.

However, within the context of the selection monad, it becomes evident
that the \(sequence\) function is unnecessarily inefficient, duplicating
work already calculated in previous steps. This paper introduces two
alternative types, namely \(K\) and \(GK\), for the selection monad. It
demonstrates that the new \(K\) type is isomorphic to the existing \(J\)
type, resolving the inefficiency issue of the monadic \(sequence\)
function. Subsequently, the \(K\) type is further generalized into the
\(GK\) type. The proposition put forth in this paper advocates for the
adoption of the \(GK\) type over the traditional \(J\) type due to its
efficiency advantages. Moreover, the \(GK\) type is argued to be more
intuitive for programming and, given its broader type, offers increased
versatility for a wide array of applications involving the selection
monad.

The upcoming section delves into the selection monad, with a particular
focus on the type: \((A \rightarrow R) \rightarrow A\) representing
selection functions \cite{escardo2010selection}. The exploration of the
\(pair\) function highlights its ability to compute a new selection
function based on criteria from two existing functions. Supported by a
practical example involving decision-making scenarios and individuals
navigating paths, this section underscores the functionality of
selection functions. An analysis of the inefficiencies in the original
\(pair\) function identifies redundant computational work. The paper's
primary contribution is outlined: an illustration and proposal for an
efficient solution to enhance the performance of the \(pair\) function.
This introductory overview sets the stage for a detailed exploration of
the selection monad and subsequent discussions on optimizations.

\section{Selection Functions}\label{selection-functions}

Consider the type for selection functions introduced by Paulo Olvia and
Martin Escardo \cite{escardo2010selection} :

\begin{code}
type J r a = (a -> r) -> a
\end{code}

Now have a look at the following example. Two individuals are walking
towards each other on the pavement. A collision is imminent. At this
juncture, each individual must decide their next move. This
decision-making process can be modeled using selection functions. The
decision they need to make is going towards the street the or wall:

\begin{code}
data Decision = Street | Wall 
\end{code}

The respective selection functions given a property function that tells
them what decision is a good one, select the best one. If there are
multiple best solutions, they select an abitrary one. And if there is no
correct one, they default to walking towards the wall.

\begin{code}
s :: J Bool Decision
s p = if p Street then Street else Wall
\end{code}

When given two selection functions, a \(pair\) function can be defined
to compute a new selection function. This resultant function selects a
pair based on the criteria established by the two given selection
functions:

\begin{code}
pair :: J r a -> J r b -> J r (a,b)
pair f g p = (a,b)
  where
      a = f (\x -> p (x, g (\y -> p (x,y))))
      b = g (\y -> p (a,y))
\end{code}

To apply the \(pair\) function, a property function \(pair\) is needed
that will judge two decisions and return \(True\) if a crash is avoided
and \(False\) otherwise.

\begin{code}
pred :: (Decision, Decision) -> Bool
pred (d1, d2) = d1 /= d2
\end{code}

The \(pair\) function, merges the two selection functions into a new one
that calculates an overall optimal decision.

\begin{haskell}
ghci> pair s s pred
(Street,Wall)
\end{haskell}

Examining how the \(pair\) function is defined reveals that the first
element \(a\) of the pair is determined by applying the initial
selection function \(f\) to a newly constructed property function.
Intuitively, selection functions can be conceptualized as entities
containing a collection of objects, waiting for a property function to
assess their underlying elements. Once equipped with a property
function, they can apply it to their elements and select an optimal one.
Considering the types assigned to selection functions, it is evident
that an initial selection function \(f\) remains in anticipation of a
property function of type \((A \rightarrow R)\) to determine an optimal
\(A\). The \(pair\) function is endowed with a property function \(p\)
of type \(((A,B) \rightarrow R)\). Through the utilization of this
property function, a property function for \(f\) can be derived by using
the second selection function \(g\) to select a corresponding \(B\) and
subsequently applying \(p\) to assess \((A,B)\) pairs as follows:
\((\lambda x \rightarrow p (x, g (\lambda y \rightarrow p (x,y))))\).
Upon the determination of an optimal \(A\), a corresponding \(B\) can
then be computed as \(g (\lambda y \rightarrow p (a,y))\). In this case,
the \(pair\) function can be conceptualized as a function that
constructs all possible combinations of the elements within the provided
selection function and subsequently identifies the overall optimal one.
It might feel intuitive to consider the following modified \(pair\)
function that seems to be more symmetric.

\begin{code}
pair' :: J r a -> J r b -> J r (a,b)
pair' f g p = (a,b)
  where
      a = f (\x -> p (x, g (\y -> p (x,y))))
      b = g (\y -> p (f (\x -> p (x,y)), y))
\end{code}

However, applying this modified \(pair'\) to our previous example this
results in a overall non optimal solution.

\begin{haskell}
ghci> pair' p1 p2 pred
(Wall,Wall)
\end{haskell}

This illustrates how the original \(pair\) function keeps track of its
first decision when determining its second element. It is noteworthy
that, in the example, achieving a satisfying outcome for both
pedestrians is only possible when they consider the direction the other
one is heading. The specific destination does not matter, as long as
they are moving in different directions. Consequently, the original
\(pair\) function can be conceived as a function that selects the
optimal solution while retaining awareness of previous solutions,
whereas our modified \(pair'\) does not. An issue with the original
\(pair\) function might have been identified by the attentive reader.
There is redundant computational work involved. Initially, all possible
pairs are constructed to determine an optimal first element \(A\), but
the corresponding \(B\) that renders it an overall optimal solution is
overlooked, resulting in only \(A\) being returned. Subsequently, the
optimal \(B\) is recalculated based on the already determined optimal
\(A\) when selecting the second element of the pair. The primary
contribution of this paper will be to illustrate and propose a solution
to this inefficiency.

\subsection{Sequence}\label{sequence}

The generalization of the pair function to accommodate a sequence of
selection functions is the initial focus of exploration. In the context
of selection functions, a \(sequence\) operation is introduced, capable
of combining a list of selection functions into a singular selection
function that, in turn, selects a list of objects:

\begin{code}
sequence :: [J r a] -> J r [a]
sequence [] p     = []
sequence (e:es) p = a : as
  where 
      a  = e (\x -> p (x : sequence es (p . (x:))))
      as = sequence es (p . (a:))
\end{code}

Here, similar to the pair function, the sequence function extracts
elements from the resulting list through the corresponding selection
functions. This extraction is achieved by applying each function to a
newly constructed property function that possesses the capability to
foresee the future, thereby constructing an optimal future based on the
currently examined element. However, a notable inefficiency persists,
exacerbating the issue observed in the pair function. During the
determination of the first element, the \(sequence\) function calculates
an optimal remainder of the list, only to overlook it and redundantly
perform the same calculation for subsequent elements. This inefficiency
in \(sequence\) warrants further investigation for potential
optimization in subsequent sections of this research paper.

\subsection{Selection monad J}\label{selection-monad-j}

The formation of a monad within the selection functions unfolds as
follows \cite{escardo2010selection}:

\begin{code}
(>>=) :: J r a -> (a -> J r b) -> J r b
(>>=) f g p = g (f (p . flip g p)) p
\end{code}

\begin{code}
return :: a -> J r a
return x p = x
\end{code}

These definitions illustrate the monadic structure inherent in selection
functions. The Haskell standard library already incorporates a built-in
function for monads, referred to as \(sequence'\), defined as:

\begin{code}
sequence' :: [J r a] -> J r [a]
sequence' []     = return []
sequence' (ma:mas) = ma >>= 
                    \x -> sequence' mas >>= 
                    \xs -> return (x:xs)
\end{code}

Notably, in the case of the selection monad, this built-in \(sequence'\)
function aligns with the earlier provided \(sequence\) implementation.
This inherent consistency further solidifies the monadic nature of
selection functions, underscoring their alignment with established
Haskell conventions.

\subsection{Illustration of Sequence in the Context of Selection
Functions}\label{illustration-of-sequence-in-the-context-of-selection-functions}

To ilustrate the application of the sequence function within the domain
of selection functions, consider a practical scenario
\cite{hartmann2022algorithm}: the task of cracking a secret password. In
this hypothetical situation, a black box property function \(p\) is
provided that returns wether the correct password is entered.
Additionally, knowledge is assumed that the password is six characters
long:

\begin{code}
p :: String -> Bool
p "secret" = True
p _        = False
\end{code}

Suppose access is available to a \(maxWith\) function, defined as:

\begin{code}
maxWith :: Ord r => [a] -> J r a
maxWith xs f = snd (maximumBy (compare `on` fst) 
                              (map (\x -> (f x , x)) xs))
\end{code}

With these resources, a selection function denoted as \(selectChar\) can
be constructed, which, given a property function that evaluates each
character, selects a single character satisfying the specified property
function:

\begin{code}
selectChar :: J Bool Char
selectChar = maxWith ['a'..'z']
\end{code}

It's worth noting that the use of maxWith is facilitated by the ordered
nature of booleans in Haskell, where \(True\) is considered greater than
\(False\). Leveraging this selection function, the sequence function can
be employed on a list comprising six identical copies of \(selectChar\)
to successfully crack the secret password. Each instance of the
selection function focuses on a specific character of the secret
password:

\begin{haskell}
ghci> sequence (replicate 6 selectChar) p
"secret"
\end{haskell}

This illustrative example not only showcases the application of the
\(sequence\) function within the domain of selection functions but also
emphasizes its utility in addressing real-world problems, such as
scenarios involving password cracking. Notably, there is no need to
explicitly specify a property function for judging individual character;
rather, this property function is constructed within the monads bind
definition, and its utilization is facilitated through the application
of the \(sequence\) function. Additionally, attention should be drawn to
the fact that this example involves redundant calculations. After
determining the first character of the secret password, the system
overlooks the prior computation of the entire password and initiates the
calculation anew for subsequent characters.

\subsection{Efficiency Issues}\label{efficiency-issues}

This inefficiency will be examined in more detail. When the \(sequence\)
function is utilized for the selection monad, an exhaustive search of
all possible combinations of the values underlying the selection
functions is executed. It is assumed that the \(minWith\) function
precisely applies the property function \(p\) once to each of its
elements. The efficiency of the \(sequence\) function is scrutinized to
determine how often the property function \(p\) is invoked during the
calculation of a solution.

Given that \(sequence\) operates as an exhaustive search resembling a
tree search with a branching factor of \(K\), the number of times the
property function \(p\) is called for a tree of depth \(n\) can be
expressed as \(T(n) = F(n) + T(n-1)\), where \(F(n) = K * T(n-1)\).
Substituting \(F(n)\) into \(T(n)\) yields
\(T(n) = K * T(n-1) + T(n-1)\). This simplifies to \(T(n) = (K + 1)^n\).
While an exhaustive search on a tree can be performed with
\(T(n) = K^n\) calls of \(p\), the \(sequence\) function duplicates some
of the work by forgetting previously computed results.

To address this specific inefficiency within the selection monad,
concerning the pair and sequence functions, two new variations of the
selection monad will be introduced. Initially, an examination of a new
specil \(K\) type will reveal its isomorphism to the selection monad
\(J\). Subsequently, an exploration of the generalization of this \(K\)
type to the \(GK\) type will enhance its intuitive usability.
Remarkably, it will be demonstrated that the \(J\) monad can be embedded
into this general \(GK\) type.

\section{Special K}\label{special-k}

The following type \(K\) is to be considered:

\begin{code}
type K r a = forall b. (a -> (r,b)) -> b
\end{code}

While selection functions of type \(J\) are in anticipation of a
property function capable of judging their underlying elements, a
similar operation is performed by the new \(K\) type. The property
function of the \(K\) type also assesses its elements by transforming
them into \(R\) values. Additionally, it converts the \(A\) into any
\(B\) and returns that \(B\) along with its judgment \(R\).

\begin{code}
pairK :: K r a -> K r b -> K r (a,b)
pairK f g p = f (\x -> 
              g (\y -> let (r, z) = p (x,y) 
                       in (r, (r,z))))
\end{code}

The previously mentioned inefficiency is now addressed by the definition
of \(pairK\). This is achieved by examining every element \(x\) in the
selection function \(f\). For each element, a corresponding result is
extracted from the second selection function \(g\). Utilizing the
additional flexibility provided by the new \(K\) type, the property
function for \(g\) is now constructed differently. Instead of merely
returning the result \(z\) along with the corresponding \(R\) value, a
duplicate of the entire result pair calculated by \(p\) is generated and
returned. As this duplicate already represents the complete solution,
the entire result for an optimal \(x\) can now be straightforwardly
yielded by \(f\), eliminating the need for additional computations.

The \(sequenceK\) for this novel \(K\) type can be defined as follows:

\begin{code}
sequenceK :: [K r a] -> K r [a]
sequenceK [] p     = p []
sequenceK (e:es) p = e (\x -> sequenceK es 
                       (\xs -> let (r,y) = p (x:xs) 
                               in (r,(r,y))))
\end{code}

This \(sequenceK\) implementation employs the same strategy as the
earlier \(pairK\) function. It essentially generates duplicates of the
entire solution pair, returning these in place of the result value. The
selection function one layer above then unpacks the result pair,
allowing the entire solution to be propagated. The efficiency issues
previously outlined are addressed by these novel \(pairK\) and
\(sequenceK\) functions. It will be further demonstrated that this \(K\)
type is isomorphic to the preceding \(J\) type. This essentially
empowers the transformation of every problem previously solved with the
\(J\) type into the world of the \(K\) type. Subsequently, the solutions
can be computed more efficiently before being transformed back to
express them in terms of \(J\).

\subsection{Special K is isomorphic to
J}\label{special-k-is-isomorphic-to-j}

To demonstrate the isomorphism between the new Special \(K\) type and
the \(J\) type, two operators are introduced for transforming from one
type to the other:

\begin{code}
j2k :: J r a -> K r a
j2k f p = snd (p (f (fst . p)))
\end{code}

When provided with a selection function \(f\) of type \(J_{R,A}\), the
\(j2k\) operator constructs an entity of type \(K\). For a given \(f\)
of type \((A \rightarrow R) \rightarrow A\) and \(p\) of type
\(\forall B. (A \rightarrow (R,B))\), the objective is to return an
entity of type \(B\). This is achieved by initially extracting an a from
\(f\) using the constructed property function \((fst \circ p)\).
Subsequently, this a is employed to apply \(p\), yielding an \((R,B)\)
pair, from which the \(B\) is obtained by applying \(snd\) to the pair.
The transformation of a selection function of type \(K\) into a
selection function of type \(J\) is accomplished as follows:

\begin{code}
k2j :: K r a -> J r a
k2j f p = f (\x -> (p x, x)) 
\end{code}

Given a selection function \(f\) of type
\(\forall B. (A \rightarrow (R,B)) \rightarrow B\) and a \(p\) of type
\((A \rightarrow R) \rightarrow A\), an \(A\) can be directly extracted
from \(f\) by constructing a property function that utilizes \(p\) to
obtain an \(R\) value while leaving the corresponding \(x\) of type
\(A\) untouched. To validate that these two operators indeed establish
an isomorphism between \(J\) and \(K\), the following equations must be
proven: \((k2j \circ j2k) f = f\) and \((j2k \circ k2j) g = g\).

\begin{proof}[J to K Embedding]\\
The equality $(k2j \circ j2k) f = f$ can be straightforwardly demonstrated by applying all the 
lambdas and the definitions of $fst$ and $snd$:
\begin{reasoning}
  \ind{(k2j \circ j2k) f}
  \equals{Apply definitions}
  \ind{(\lambda g\:p_2 \rightarrow g (\lambda x \rightarrow (p_2\:x, x))) (\lambda p_1 \rightarrow snd (p_1 (f (fst \circ p_1))))}
  \equals{Simplify}
  \ind{f}
\end{reasoning}
\end{proof}

This proof involves a direct application of lambda expressions and the
definitions of \(fst\) and \(snd\) for simplification. To facilitate the
proof of the second isomorphism, we initially introduce the free theorem
for the special \(K\) type \cite{wadler1989theorems}:

\begin{theorem}[Free Theorem for K]\\
Given the following functions with their corresponding types:
\begin{reasoning}
$g : K_{R,A}$\\
$h : B_1 \rightarrow B_2$\\
$p : A \rightarrow (R, B_1)$\\
\end{reasoning}
We have:
\[h (g\:p) = g ((id *** h) \circ p)\]
\end{theorem}

The free theorem essentially asserts that a function \(h\) of type
\(B_1 \rightarrow B_2\), when applied to the result of a selection
function, can also be incorporated into the property function and
applied to each individual element. This follows from the generalized
type of \(K\), where the only means of generating \(B_1\) values is
through the application of \(p\). Consequently, it becomes
inconsequential whether h is applied to the final result or to each
individual intermediate result. With the free theorem for \(K\), the
remaining portion of the isomorphism can now be demonstrated as follows:

\begin{proof}[K to J Embedding]\\
The equality $(j2k \circ k2j) g = g$ is established through the following steps:\\
\begin{reasoning}
  \ind{(j2k \circ k2j) g}
  \equals{Apply definitions and simplify}
  \ind{\lambda p \rightarrow snd (p (g (\lambda x \rightarrow ((fst \circ p) x, x))))}
  \equals{Free Theorem for $K$}
  \ind{\lambda p \rightarrow g (\lambda x \rightarrow ((fst \circ p) x, (snd \circ p) x))}
  \equals{Simplify}
  \ind{g}
\end{reasoning}
\end{proof}

The monad definitions and \(sequence\) definition for the new \(K\) type
can be derived from the isomorphism. While the desired performance
improvements are achieved by the definition of \(K\), significant data
structure copying is required, only to be deconstructed and discarded at
a higher layer. This process significantly complicates the associated
definitions for \(sequence\) and \(pair\), rendering them challenging to
handle and lacking in intuitiveness. Introducing another type, \(GK\),
that returns the entire tuple rather than just the result value seems
more intuitive. This exploration is detailed in the following section,
where similar performance improvements are observed with \(GK\) while
the definitions become more straightforward. This approach also
eliminates the need for unnecessary copying of data. However, it is
revealed that \(GK\) is not isomorphic to \(J\) and \(K\); instead, they
can be embedded into \(GK\). Conversely, we will explore a specific
precondition under which \(GK\) can be embedded into \(J\) or \(K\).

\section{General K}\label{general-k}

Consider the more general type \(GK\), derived from the previous special
\(K\) type:

\begin{code}
type GK r a = forall b. (a -> (r,b)) -> (r,b)
\end{code}

Unlike its predecessor, \(GK\) returns the entire pair produced by the
property function, rather than just the result value. The implementation
of \(pairGK\) for the new \(GK\) type no longer necessitates the
creation of a copy of the data structure. It suffices to return the
result of the property function's application to the complete pair:

\begin{code}
pairGK :: GK r a -> GK r b -> GK r (a,b)
pairGK f g p = f (\x -> g (\y -> p (x,y)))
\end{code}

In terms of readability, this definition of \(pairGK\) is significantly
more concise, conveying the essence of the \(pair\) function without
unnecessary boilerplate code. For every element \(x\) of type \(A\)
within \(f\), all \(y\) of type \(B\) within \(g\) are inspected and
judged by the given property function \(p\). The resulting pair
selection function returns the optimal pair of \((A,B)\) values
according to the provided property function. Furthermore, we define
\(sequenceGK\) as follows:

\begin{code}
sequenceGK :: [GK r a] -> GK r [a]
sequenceGK [e] p    = e (\x  -> p [x])
sequenceGK (e:es) p = e (\x  -> sequenceGK es 
                        (\xs -> p (x:xs)))
\end{code}

Following a similar pattern, this \(sequenceGK\) function builds all
possible futures for each element within \(e\). Once an optimal list of
elements is found, this list is simply returned along with the
corresponding \(R\) value.

\subsection{Relationship to J and Special
K}\label{relationship-to-j-and-special-k}

With the following operators, selection functions of type \(K\) can be
embedded into \(GK\).

\ignore{

> gk2k :: forall r a b. ((a -> (r,b)) -> (r,b)) -> ((a -> (r,b)) -> b)
> gk2k f = snd . f

}

\begin{haskell}
gk2k :: GK r a -> K r a 
gk2k f = snd . f
\end{haskell}

\begin{code}
k2gk :: K r a -> GK r a
k2gk f p = f (\x -> let (r,y) = p x in (r, (r,y)))
\end{code}

Similar to the free theorem for the \(K\) type, it is equally possible
to derive the free theorem for the new \(GK\) type
\cite{wadler1989theorems}:

\begin{theorem}[Free Theorem for GK]\\
Given the following functions with thier corresponding types:
\begin{reasoning}
  $g : GK_{R,A}$\\
  $f : B_1 \rightarrow B_2$\\
  $p : A \rightarrow (R, B_1)$\\
\end{reasoning}
We have:
\[((id *** f) \circ g) p = g ((id *** f) \circ p)\]
\end{theorem}

This theorem essentially conveys the same concept as the free theorem
for \(K\). It asserts that given a function \(f\) applied to the result
of a selection function, the order of application, whether at the final
stage to the ultimate result or inside the property function of the
selection function, does not impact the outcome. However, this
formulation now accommodates the fact that the \(GK\) type also returns
the \(R\) value.

With the free theorem for \(GK\), we can establish that selection
functions of type \(K\) can be seamlessly embedded into \(GK\):

\begin{proof}[K to GK Embedding]\\
The equality $(k2gk \circ gk2k) f = f$ is established through the following steps:\\
Assuming: $f : K_{R,A}$
\begin{reasoning}
  \ind{(gk2k \circ k2gk) f}
  \equals{Definitions and rewrite}
  \ind{\lambda p \rightarrow (snd \circ f) (\lambda x \rightarrow \text{ let }(r,y) = p\:x\text{ in } (r, (r,y)))}
  \equals{Free theorem of $GK$}
  \ind{\lambda p \rightarrow f (\lambda x \rightarrow \text{ let }(r,y) = p\:x \text{ in }(r, snd (r,y)))}
  \equals{Simplify}
  \ind{f}
\end{reasoning}
\end{proof}

Embedding \(K\) selection functions into the new \(GK\) type introduces
a slightly more intricate process. The key requirement is to ensure that
the function \(g\) does not alter the \(R\) value subsequent to applying
\(p\) to its elements. Hence:

\begin{proof}[GK to K Embedding]\\
The equality $(k2gk \circ gk2k) g = g$ is established through the following steps:\\
Assuming that for:
\begin{reasoning}
  $g : GK_{R,A} $\\
  $\forall p : (\forall B . (A \rightarrow (R,B))), \exists x : A \text{ such that: } g\:p = p\:x$
\end{reasoning}
We can reason:
\begin{reasoning}
\ind{(k2gk \circ gk2k) g}
  \equals{Definitions and rewrite}
  \ind{\lambda p \rightarrow snd (g(\lambda x \rightarrow \text{ let }(r,y) = p\:x \text{ in }(r, (r,y))))}
  \equals{Assumption}
  \ind{\lambda p \rightarrow snd (\exists x. \text{ let }(r,y) = p\:x\text{ in } (r, (r,y)))}
  \equals{Exists commutes}
  \ind{\lambda p \rightarrow \exists x.\text{ let }(r,y) = p\:x\text{ in }snd (r, (r,y))}
  \equals{Assumption}
  \ind{\lambda p \rightarrow g (\lambda x \rightarrow \text{ let }(r,y) = p\:x \text{ in }snd (r, (r,y)))}
  \equals{Simplify}
  \ind{g}
\end{reasoning}
\end{proof}

TODO: - elaborate on the precondition for the embedding -
counterexamples to ilustrate what precondition means and why we want it

\section{GK forms a monad}\label{gk-forms-a-monad}

The formation of the monad for \(GK\) follows a straightforward
definition:

\begin{code}
bindGK :: GK r a -> (a -> GK r b) -> GK r b
bindGK e f p = e (\x -> f x p)
\end{code}

In this context, given a selection function \(e\) of type \(GK_{R A}\),
a function \(f\) of type \(A \rightarrow GK_{R,A}\), and a property
function \(p\) of type \(\forall C. (B \rightarrow (R,C))\), the outcome
of type \((R,C)\) is assembled through the utilization of \(e\). Each
element \(x\) of type \(A\) underlying \(e\) undergoes assessment by
applying \(f\). This process yields a pair consisting of the \(R\)
value, which serves as the basis for judgment, and the result value of
type \(C\). As the pair is already of the correct type, a
straightforward return suffices.

The return for the \(GK\) type is defined as follows:

\begin{code}
returnGK :: a -> GK r a
returnGK x p = p x
\end{code}

The proofs substantiating the monad laws are annexed in the appendix.

Exploring the alignment of these monad definitions with those of \(J\)
or \(K\), respectively, is our next objective. The aim is to ensure that
the behavior of the \(GK\) monad aligns with that of the \(J\) and \(K\)
monads.

To derive the monad definitions from the embedding operators, it is
imperative to introduce the following two theorems:

\begin{theorem}[Theorem 1]\\
Given:
\begin{reasoning}
$f : (R,B_1) \rightarrow (R,B_2)$\\
$g : GK_{R,A}$\\
$p : A \rightarrow (R,B_1)$
\end{reasoning}
We have:
\[fst \circ f \circ p = fst \circ p\ \implies (f \circ g)p = g (f \circ p)\]
\end{theorem}

This theorem asserts that when we have a function \(f\) applied to the
outcome of a selection function of type \(GK_{R,A}\), we can similarly
apply \(f\) to each underlying element of \(GK{R,A}\) within the
property function. This is contingent upon the condition that \(f\) only
modifies the \(B\) value and does not alter the \(R\) value.

\newpage
\begin{proof}[Theorem 1]\\
Assuming that for:
\begin{reasoning}
$(1)\:f : (R,B_1) \rightarrow (R,B_2),g : GK_{R,A}, p : A \rightarrow (R,B_1)$\\
$(2)\:\forall p : (\forall B . (A \rightarrow (R,B))), \exists x : A \text{ such that } g\:p = p\:x$\\
$(3)\:fst \circ f \circ p = fst \circ p$
\end{reasoning}
We can reason:
\begin{reasoning}
  \ind{f (g\:p)}
  \equals{Assumption (2)}
  \ind{\exists x . f (p\:x)}
  \equals{Rewrite as tuple}
  \ind{\exists x .((fst \circ f \circ p) x, (snd \circ f \circ p) x)}
  \equals{Assumption (3)}
  \ind{\exists x .((fst \circ p ) x , (snd \circ f \circ p) x)}
  \equals{Rewrite as lambda}
  \ind{\exists x .(\lambda (r,y) \rightarrow (r, (snd \circ f) (r, y))) p\:x}
  \equals{Assumption (2)}
  \ind{(\lambda (r,y) \rightarrow (r, (snd \circ f) (r, y))) g\:p}
  \equals{Free Theorem for $GK$}
  \ind{g ((\lambda (r,y) \rightarrow  (r, (snd \circ f) (r,y))) \circ p) }
  \equals{Rewrite}
  \ind{g (\lambda x \rightarrow ((fst \circ p) x,  (snd \circ f \circ p) x))}
  \equals{Assumption (3)}
  \ind{g (\lambda x \rightarrow ((fst \circ f \circ p) x, (snd \circ f \circ p) x))}
  \equals{Simplify}
  \ind{g (f \circ p)}
\end{reasoning}
\end{proof}

To further simplify the calculation we also introduce the following
theorem:

\begin{theorem}[Theorem 2]\\
If $q$ does apply $p$ to get the $R$ value but keeps the original value, and we then use that 
original value to compute the $(R,Z)$ values with $p$ we can call $g$ with $p$ directly.\\
Given:
\begin{reasoning}
$p :: A \rightarrow (R,B)$\\
$g :: K_{R,A}$\\
\end{reasoning}
We have:
\[(p \circ snd) (g\:q) = g\:p \text{ where } q = \lambda x \rightarrow ((fst \circ p) x, x)\]
\end{theorem}

And we can proof Theorem 2 by utilising Theorem 1.

\begin{proof}[Theorem 2]\\
\begin{reasoning}
  \ind{(p \circ snd) (g\:q)}
  \equals{Definition of $q$}
  \ind{(p \circ snd) (g\:(\lambda x \rightarrow ((fst \circ p) x, x)))}
  \equals{Theorem 1}
  \ind{g (\lambda x \rightarrow (p \circ snd) ((fst \circ p) x, x))}
  \equals{Simplify}
  \ind{g\:p}
\end{reasoning}
$\iff$
\begin{reasoning}
  \ind{(fst \circ p \circ snd) (\lambda x \rightarrow ((fst \circ p) x, x))}
  \equals{Simplify}
  \ind{\lambda y \rightarrow (fst ( p (snd ( (\lambda x \rightarrow ((fst \circ p) x, x)) y))))}
  \equals{Simplify}
  \ind{\lambda y \rightarrow (fst(p(snd ((fst \circ p) y, y) )))}
  \equals{Simplify}
  \ind{\lambda x \rightarrow (fst \circ p) x}
  \equals{Simplify}
  \ind{fst \circ (\lambda x \rightarrow ((fst \circ p) x, x))}
\end{reasoning}
\end{proof}

-- TODO: Give an intuition what these theorems mean

Now, consider the following two operators that transform between \(GK\)
selection functions and \(J\) selection functions:

\begin{code}
j2gk :: J r x -> GK r x
j2gk f p = p (f (fst . p))
\end{code}

\begin{code}
gk2j :: GK r x -> J r x
gk2j f p = snd (f (\x -> (p x, x)))
\end{code}

We can calculate the bind implementation for \(GK\) with the \(j2gk\)
and \(gk2j\) operators and the previusly introduced theorems:

\begin{proof}[GK Monad behaves similar to J]\\
\begin{reasoning}
  \ind{j2gk (gk2j\:f >>= gk2j \circ g)}
  \equals{Definition of $J_{>>=}$}
  \ind{j2gk ((\lambda f\:g\:p \rightarrow g (f (p \circ flip\:g\:p)) p) (gk2j\:f) (gk2j \circ g))}
  \equals{simplify}
  \ind{j2gk (\lambda p \rightarrow gk2j (g (gk2j\:f (p \circ (\lambda x \rightarrow gk2j (g\:x) p)))) p)}
  \equals{Definition of j2k and rewrite}
  \ind{\lambda p \rightarrow p (gk2j (g (gk2j f (\lambda x \rightarrow fst ((p \circ snd) ((g\:x) (\lambda x \rightarrow ((fst \circ p) x, x))))))) (fst \circ p))}
  \equals{Theorem 1}
  \ind{\lambda p \rightarrow p (gk2j (g (gk2j f (\lambda x \rightarrow fst (((g\:x) (\lambda x \rightarrow (p \circ snd) ((fst \circ p) x, x))))))) (fst \circ p))}
  \equals{Definition of j2gk and rewrite}
  \ind{\lambda p \rightarrow p (snd (g (snd (f (\lambda x \rightarrow (fst (g\:x\:p), x)))) (\lambda x \rightarrow ((fst \circ p) x, x))))}
  \equals{Theorem 2}
  \ind{\lambda p \rightarrow g (snd (f (\lambda x \rightarrow (fst (g\:x\:p), x)))) p}
  \equals{Rewrite}
  \ind{\lambda p \rightarrow (\lambda y \rightarrow g (snd\:y) p) (f (\lambda x \rightarrow (fst (g\:x\:p), x)))}
  \equals{Theorem 1}
  \ind{\lambda p \rightarrow f ((\lambda y \rightarrow g (snd\:y) p) \circ (\lambda x \rightarrow (fst (g\:x\:p), x))) }
  \equals{Simplify}
  \ind{\lambda p \rightarrow f (\lambda x \rightarrow g\:x\:p)}
\end{reasoning}
\end{proof}

This shows that all \(GK\) selection functions sattisfing the
precodition behave the same when transforemd to \(K\) or \(J\) selection
functions.

\begin{itemize}
\tightlist
\item
  TODO: ilustrate how nice it is to deal with
\end{itemize}

\section{Performance analisys}\label{performance-analisys}

\begin{itemize}
\tightlist
\item
  give some perfomance analysis examples that ilustrate improvement
\item
  Done by an example and use trace to count calls of P
\end{itemize}

\section{Related work}\label{related-work}

\begin{itemize}
\item
  J was researched in the context of Sequential games, but slowly found
  its way to other applications
\item
  It can also be used for greedy algorythms, however this performance
  optimisation does not apply in this case
\item
  But greedy algorythms can also be represented with the new General
  selection monad
\end{itemize}

\section{Outlook and future work}\label{outlook-and-future-work}

\begin{itemize}
\tightlist
\item
  Need to investigate further whats possible with the more general type
\item
  Alpha beta pruning as next step of my work
\end{itemize}

\section{Conclusion}\label{conclusion}

\begin{itemize}
\tightlist
\item
  We should use General K istead of J because more useful and more
  intuitive once understood
\item
  performance improvements are useful
\item
  monad, pair, and sequence implementation much more intuitive and
  useful
\end{itemize}

%
% ---- Bibliography ----
%
% BibTeX users should specify bibliography style 'splncs04'.
% References will then be sorted and formatted in the correct style.
%
% \bibliographystyle{splncs04}
% \bibliography{mybibliography}
%
\bibliographystyle{splncs04}
\bibliography{bib}

\newpage
\section*{Appendix}
\appendix
\subsection*{Proof Monad Laws for GK}\label{GK-monad-laws}
\begin{proof}[Left identity]
\begin{haskell}
return a >>= h
= (flip ($)) a >>= h
= (\p -> p a) >>= h
= \p' -> (\p -> p a) ((flip h) p')
= \p' -> ((flip h) p') a
= \p' -> h a p'
= h a
\end{haskell}
\end{proof}
\begin{proof}[Right identity]
\begin{haskell}
m >>= return
= \p -> m ((flip return) p)
= \p -> m ((flip (flip ($))) p)
= \p -> m (($) p)
= \p -> m p
= m
\end{haskell}
\end{proof}
\begin{proof}[Associativity]
\begin{haskell}
(m >>= g) >>= h
= \p -> (m >>= g) ((flip h) p)
= \p -> (\p' -> m ((flip g) p')) ((flip h) p)
= \p -> (m ((flip g) ((flip h) p)))
= \p -> m ((\y x -> g x y) ((flip h) p))
= \p -> m ((\x -> g x ((flip h) p)))
= \p -> m ((\p' x -> (g x) ((flip h) p')) p)
= \p -> m ((flip (\x p' -> (g x) ((flip h) p'))) p)
= \p -> m ((flip (\x -> (\p' -> (g x) ((flip h) p')))) p)
= \p -> m ((flip (\x -> g x >>= h)) p)
= m >>= (\x -> g x >>= h)
\end{haskell}
\end{proof}
\end{document}
